{\color{gray}\hrule}
\begin{center}
\section{Conclusion}
\bigskip
\end{center}
{\color{gray}\hrule}

\begin{multicols}{2}
  According to the slope (see result \ref{eq:results:slope}) obtained in section \ref{sec:results}, measured precession period corresponds one-to-one with the calculated precession period. This means that the precession frequency changes linearly with the applied torque. Besides, for different torques and initial spinning frequencies of the rotor axis, equality $\boldsymbol\tau = \frac{d\mathbf{L}}{dt}$ is observed. This implies, that even for complex rotationary systems like the gyroscope, classical mechanics of rotations apply. Lastly, in absence of a net external toruqe, angular momentum is conserved as discussed in section \ref{sec:discussion:no}.
\end{multicols}
