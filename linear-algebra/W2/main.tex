\documentclass{article}

\usepackage{float}
\usepackage{amsmath}
\usepackage{amsfonts}
\usepackage{amssymb}
\usepackage{mhchem}
\usepackage{graphicx}
\usepackage[margin=1in]{geometry}

\usepackage[backend=biber, style=alphabetic, sorting=ynt]{biblatex}

% for augmented matrices
\newenvironment{amatrix}[1]{\left(\begin{array}{@{}*{#1}{c}|c@{}}}{\end{array}\right)}

\title{Week 2 - Homework 1}
\author{Artur Topal, S5942128}
\date{\today}

\begin{document}

\maketitle

\begin{center}
  \textbf{TA}: Teodora Neaga, group 4
\end{center}

\pagebreak

\section{ Problem 1 } 

The augmented matrix for the system is $\begin{amatrix}{2} 2 & 1 & 7 \\  1 & -3 & -8 \end{amatrix}$. Put it in the RREF\footnote{Reduced row echelon form.}.

\begin{equation*}
  \begin{amatrix}{2} 2 & 1 & 7 \\  1 & -3 & -8 \end{amatrix} =
  \begin{amatrix}{2} 1 & -3 & -8 \\  2 & 1 & 7 \end{amatrix} =
  \begin{amatrix}{2} 1 & -3 & -8 \\  2 - 2 \cdot 1 & 1 - 2 \cdot (-3) & 7 - 2 \cdot (-8) \end{amatrix} =
  \begin{amatrix}{2} 1 & -3 & -8 \\  0 & 7 & 23 \end{amatrix} =
  \begin{amatrix}{2} 1 & -3 & -8 \\  0 & 1 & \frac{23}{7} \end{amatrix} = 
\end{equation*}
\begin{equation*}
  \begin{amatrix}{2} 1 & -3 + 3 \cdot 1 & -8 + 3 \cdot \frac{23}{7} \\  0 & 1 & \frac{23}{7} \end{amatrix} =
  \begin{amatrix}{2} 1 & 0 & \frac{13}{7} \\ 0 & 1 & \frac{23}{7} \end{amatrix}
\end{equation*}

Therefore, $x_{1} = \frac{13}{7}$ and $x_{2} = \frac{23}{7}$.

\section{ Problem 2 }

The same as in Problem 1.

\begin{equation*}
  \begin{amatrix}{3}
    2 & 3 & 1 & 10 \\
    4 & -1 & 5 & 5 \\
    1 & 2 & -3 & -4
  \end{amatrix} =
  \begin{amatrix}{3}
    1 & 2 & -3 & -4 \\
    4 - 4 \cdot 1 & -1 - 4 \cdot 2 & 5 - 4 \cdot (-3) & 5 - 4 \cdot (-4) \\
    2 - 2 \cdot 1 & 3 - 2 \cdot 2 & 1 - 2 \cdot (-3) & 10 - 2 \cdot (-4)
  \end{amatrix} =
  \begin{amatrix}{3}
    1 & 2 & -3 & -4 \\
    0 & -9 & 17 & 21 \\
    0 & -1 & 7 & 18
  \end{amatrix} = 
\end{equation*}
\begin{equation*}
  \begin{amatrix}{3}
    1 & 2 & -3 & -4 \\
    0 & 1 & -7 & -18 \\
    0 & -9 & 17 & 21
  \end{amatrix} =
  \begin{amatrix}{3}
    1 & 2 & -3 & -4 \\
    0 & 1 & -7 & -18 \\
    0 & -9 + 9 \cdot 1 & 17 + 9 \cdot (-7) & 21 + 9 \cdot (-18)
  \end{amatrix} =
  \begin{amatrix}{3}
    1 & 2 & -3 & -4 \\
    0 & 1 & -7 & -18 \\
    0 & 0 & -46 & -141
  \end{amatrix} =
\end{equation*}
\begin{equation*}
  \begin{amatrix}{3}
    1 & 2 & -3 & -4 \\
    0 & 1 & -7 & -18 \\
    0 & 0 & 1 & \frac{141}{46}
  \end{amatrix} =
  \begin{amatrix}{3}
    1 & 2 & -3 + 3 \cdot 1 & -4 + 3 \cdot \frac{141}{46} \\
    0 & 1 & -7 + 7 \cdot 1 & -18 + 7 \cdot \frac{141}{46} \\
    0 & 0 & 1 & \frac{141}{46}
  \end{amatrix} =
  \begin{amatrix}{3}
    1 & 2 & 0 & \frac{239}{46} \\
    0 & 1 & 0 & \frac{159}{46} \\
    0 & 0 & 1 & \frac{141}{46}
  \end{amatrix} =
  \begin{amatrix}{3}
    1 & 2 - 2 \cdot 1 & 0 & \frac{239}{46} - 2 \cdot \frac{159}{46} \\
    0 & 1 & 0 & \frac{159}{46} \\
    0 & 0 & 1 & \frac{141}{46}
  \end{amatrix} =
\end{equation*}
\begin{equation*}
  \begin{amatrix}{3}
    1 & 0 & 0 & -\frac{79}{46} \\
    0 & 1 & 0 & \frac{159}{46} \\
    0 & 0 & 1 & \frac{141}{46}
  \end{amatrix}
\end{equation*}

Therefore, $x_{1} = -\frac{79}{46}$, $x_{2} = \frac{159}{46}$, and $x_{3} = \frac{141}{46}$.

\section{Problem 3}

First, reduce the augmented matrix.

\begin{equation*}
  \begin{amatrix}{3}
    -3 & 1 & -1 & b \\
    -2 & 4 & a  & 3 \\
    2 & 1 & 2 & 3
  \end{amatrix} = 
  \begin{amatrix}{3}
    -1 & 2 & 1 & b + 3 \\
    0 & 5 & a + 2  & 6 \\
     2 & 1 & 2  & 3
   \end{amatrix} =
   \begin{amatrix}{3}
    1 & -2 & -1 & -b-3 \\
    1 & \frac{1}{2} & 1  & \frac{3}{2} \\
    0 & 5 & a + 2  & 6
  \end{amatrix} =
  \begin{amatrix}{3}
    1 & -2 & -1 & -b-3 \\
    0 & \frac{5}{2} & 2  & b + \frac{9}{2} \\
    0 & 5 & a + 2  & 6
  \end{amatrix} =
\end{equation*}

\begin{equation*}
  \begin{amatrix}{3}
    1 & -2 & -1 & -b-3 \\
    0 & 1 & \frac{a+2}{5}  & \frac{6}{5} \\
    0 & \frac{5}{2} & 2  & b + \frac{9}{2}
  \end{amatrix} =
  \begin{amatrix}{3}
    1 & -2 & -1 & -b-3 \\
    0 & 1 & \frac{a+2}{5}  & \frac{6}{5} \\
    0 & 0 & \frac{2-a}{2}& b + \frac{3}{2}
  \end{amatrix} =
  \begin{amatrix}{3}
    1 & -2 & -1 & -b-3 \\
    0 & 1 & \frac{4}{5}  & \frac{6}{5} + \frac{2}{5} \cdot (b + \frac{3}{2}) \\
    0 & 0 & \frac{2-a}{2}& b + \frac{3}{2}
  \end{amatrix} =
\end{equation*}
\begin{equation*}
  \begin{amatrix}{3}
    1 & -2 & -1 & -b-3 \\
    0 & 1 & \frac{4}{5}  & \frac{2b + 9}{5} \\
    0 & 0 & \frac{2-a}{2}& b + \frac{3}{2}
  \end{amatrix} = 
  \begin{amatrix}{3}
    1 & -2 & -1 & -b-3 \\
    0 & 1 & \frac{4}{5}  & \frac{2b + 9}{5} \\
    0 & 0 & a - 2& - 2b - 3
  \end{amatrix}
\end{equation*}

If $a \ne 2$, then $x_{3}$ is uniquely determined by $\frac{-2b-3}{a - 2}$. Therefore, the system has exactly one solution.

Now, assume $a = 2$. Then, the third row suggests $0x_{3} = -2b - 3$. If $-2b - 3 \ne 0$ (so, $b \ne -\frac{3}{2}$), then $0x_{3} \ne 0$ which is a contradiction. Otherwise, if $b = -\frac{3}{2}$, then $0x_{3} = 0$ which is correct for $\forall x_{3} \in \mathbb{R}$ making $x_{3}$ a free variable. So, the system has infinite number of solutions.

To summarize:

a. $a \in \mathbb{R} \setminus \{ 2 \} \land b \in \mathbb{R}$, then \textbf{1} solution.

b. $a = 2 \land b = -\frac{3}{2}$, then \textbf{infinite} solutions.

c. $a = 2 \land b \in \mathbb{R} \setminus \{ -\frac{3}{2} \}$, then \textbf{no} solutions.

\section {Problem 4} \textbf{a.} Row-reduce the matrix.

\begin{equation*}
  \begin{amatrix}{3}
    2 & 1 & 5 & 10 \\
    1 & -3 & -1 & -2 \\
    4 & -2 & 6 & 12
  \end{amatrix} =
  \begin{amatrix}{3}
    1 & -3 & -1 & -2 \\
    2 & 1 & 5 & 10 \\
    4 & -2 & 6 & 12
  \end{amatrix}
  \xrightarrow{(R_{3} \leftarrow R_{3} - 2R_{2})}
  \begin{amatrix}{3}
    1 & -3 & -1 & -2 \\
    2 & 1 & 5 & 10 \\
    0 & -4 & -4 & -8
  \end{amatrix}
  \xrightarrow{(R_{3} \leftarrow R_{3} / (-4)) \land (R_{2} \leftarrow R_{2} - 2R_{1})}
\end{equation*}
\begin{equation*} 
  \begin{amatrix}{3}
    1 & -3 & -1 & -2 \\
    0 & 7 & 7 & 14 \\
    0 & 1 & 1 & 2
  \end{amatrix}
  \xrightarrow{(R_{2} \leftarrow R_{2} / 7)}
  \begin{amatrix}{3}
    1 & -3 & -1 & -2 \\
    0 & 1 & 1 & 2 \\
    0 & 1 & 1 & 2
  \end{amatrix}
  \xrightarrow{(R_{1} \leftarrow R_{1} + 3R_{2})}
  \begin{amatrix}{3}
    1 & 0 & 2 & 4 \\
    0 & 1 & 1 & 2 \\
    0 & 1 & 1 & 2
  \end{amatrix} =
   \begin{amatrix}{3}
    1 & 0 & 2 & 4 \\
    0 & 1 & 1 & 2 \\
    0 & 0 & 0 & 0
  \end{amatrix}
\end{equation*}

The reduced matrix is $\begin{amatrix}{3}
    1 & 0 & 2 & 4 \\
    0 & 1 & 1 & 2 \\
    0 & 0 & 0 & 0
  \end{amatrix}$.

\textbf{b.} Given matrix:

\begin{equation*}
  \begin{cases}
    2x_{1} + x_{2} + 5x_{3} = 10 \\
    x_{1} -3x_{2} - x_{3} = -2 \\
    4x_{1} - 2x_{2} + 6x_{3} = 12
  \end{cases}
\end{equation*}

Row-reduced matrix:
\begin{equation*}
  \begin{cases}
    x_{1} + 2x_{3} = 4 \\
    x_{2} + x_{3} = 2
  \end{cases}
\end{equation*}

\textbf{c} $x_{3} := \alpha$ is a free variable. The solution is readily visible from the system of equations: $x_{1} = 4 - 2\alpha$, $x_{2} = 2 - \alpha$, and $\alpha \in \mathbb{R}$.

\section{Problem 5}

\begin{equation*}
  \begin{amatrix}{4}
    1 & 4 & 3 & -1 & 5 \\
    1 & -2 & 1 & 2 & 6 \\
    4 & 1 & 6 & 5 & 9
  \end{amatrix}
  \xrightarrow{(R_{3} \leftarrow R_{3} - 4R_{1} ) \land ( R_{2} \leftarrow R_{2} - R_{1} )}  
  \begin{amatrix}{4}
    1 & 4 & 3 & -1 & 5 \\
    0 & -6 & -2 & 3 & 1 \\
    0 & -15 & -6 & 9 & -11
  \end{amatrix}
  \xrightarrow{( R_{2} \leftarrow R_{2} / (-6) )}
\end{equation*}
\begin{equation*}
  \begin{amatrix}{4}
    1 & 4 & 3 & -1 & 5 \\
    0 & 1 & 1/3 & -1/2 & -1/6 \\
    0 & -15 & -6 & 9 & -11
  \end{amatrix}
  \xrightarrow{( R_{3} \leftarrow (-1) \cdot (R_{3} + 15R_{2}) )}
  \begin{amatrix}{4}
    1 & 4 & 3 & -1 & 5 \\
    0 & 1 & 1/3 & -1/2 & -1/6 \\
    0 & 0 & 1  & -3/2 & 27/2
  \end{amatrix}
\end{equation*}

$x_{4} := \alpha \in \mathbb{R}$ is a free variable. Therefore, $x_{3} = \frac{27}{2} + \frac{3}{2}\alpha$. Substituting $x_{3}$ and $x_{4}$ into the second row, we get $x_{2} = -\frac{1}{6} + \frac{\alpha}{2} - \frac{1}{3} x_{3} = -\frac{1}{6} + \frac{\alpha}{2} - \frac{1}{3} (\frac{27}{2} + \frac{3}{2}\alpha) = -\frac{1}{6} + \frac{\alpha}{2} - \frac{9}{2} - \frac{\alpha}{2} = -\frac{28}{6} = -\frac{14}{3}$. Now, substitute $x_{2}, x_{3}, x_{4}$ into the first row. We obtain: $x_{1} = 5 + x_{4} - 3x_{3} - 4x_{2} = 5 + \alpha - 3 \cdot ( \frac{27}{2} + \frac{3}{2}\alpha) - 4 \cdot \left( \frac{-14}{3} \right) = -\frac{101 + 21\alpha}{6}$

Therefore, the solution is

$x_{4} := \alpha \in \mathbb{R}$ (free variable)

$x_{3} := 13.5 + 1.5\alpha$

$x_{2} := -14/3$ (fixed variable)

$x_{1} := -\frac{101 + 21\alpha}{6}$

\section{Problem 6}

\textbf{a.} The matrix, $A$, of the system is given below and its determinant is found by a cofactor expansion along the first row.

\begin{equation*}
  \begin{vmatrix}
    -3 & 1 & -1 \\
    -2 & 4 & a  \\
    2 & 1 & 2 
  \end{vmatrix} =
  -3
  \begin{vmatrix}
    4 & a \\
    1 & 2
  \end{vmatrix}
  -\begin{vmatrix}
    -2 & a \\
    2 & 2
  \end{vmatrix}
  - \begin{vmatrix}
    -2 & 4 \\
    2 & 1
    \end{vmatrix} = -3 \cdot (8 - a) - (-4 - 2a) - (-2 - 8) = -24 + 3a + 4 + 2a + 2 + 8 = 5a - 10
\end{equation*}

Therefore, $det(A) = 5a - 10$. Consider the system $A\mathbf{x} = \mathbf{b} $, where $\mathbf{b}^{T} = \begin{pmatrix} b & 3 & 3 \end{pmatrix}$. If $det(A) \ne 0$, then A is nonsingular and, thus, its inverse, $A^{-1}$, exists, resulting in $\mathbf{x} = A^{-1}\mathbf{b}$ (in other words, a non-trivial and unique solution). Therefore, the determinant must be nonzero: $5a - 10 \ne 0 \Rightarrow a \ne 2$. This is the same result as obtained in problem 3.

\textbf{b.} Use Guass-Jordan elimination to find the inverse of $A$ for $\forall a \ne 2$.

\begin{equation*}
  \begin{pmatrix}
    -3 & 1 & -1 & 1 & 0 & 0 \\
    -2 & 4 &  a & 0 & 1 & 0 \\
     2 & 1 &  2 & 0 & 0 & 1 
   \end{pmatrix}
   \xrightarrow{(R_{1} \leftarrow (-1) \cdot (R_{1} + R_{3}))}
  \begin{pmatrix}
    1 & -2 & -1 & -1 & 0 & -1 \\
    -2 & 4 &  a & 0 & 1 & 0 \\
     2 & 1 &  2 & 0 & 0 & 1 
   \end{pmatrix}
   \xrightarrow{(R_{3} \leftarrow R_{3} + R_{2})}
 \end{equation*}
 
\begin{equation*}
  \begin{pmatrix}
    1 & -2 & -1 & -1 & 0 & -1 \\
    -2 & 4 &  a & 0 & 1 & 0 \\
     0 & 5 &  2 + a & 0 & 1 & 1
  \end{pmatrix}
  \xrightarrow{(R_{2} \leftarrow R_{2} + 2R_{1} \land P_{23})}
  \begin{pmatrix}
    1 & -2 & -1 & -1 & 0 & -1 \\
   0 & 5 &  2 + a & 0 & 1 & 1 \\
    0 & 0 &  a - 2 & -2 & 1 & -2 
  \end{pmatrix}
  \xrightarrow{((R_{3} \leftarrow R_{3} / (a-2)) \land (R_{2} \leftarrow R_{2} / 5))}
\end{equation*}

\begin{equation*}
  \begin{pmatrix}
    1 & -2 & -1 & -1 & 0 & -1 \\
    0 & 1 &  \frac{2 + a}{5} & 0 & 1/5 & 1/5 \\
    0 & 0 &  1 & \frac{-2}{a-2} & \frac{1}{a-2} & \frac{-2}{a-2} 
  \end{pmatrix}
  \xrightarrow{(R_{2} \leftarrow R_{2} - \frac{2+a}{5} R_{3})}
  \begin{pmatrix}
    1 & -2 & -1 & -1 & 0 & -1 \\
    0 & 1 &  0 & \frac{2(2+a)}{5(a-2)} & -\frac{4}{5(a-2)} & \frac{3a+2}{5(a-2)} \\
    0 & 0 &  1 & \frac{-2}{a-2} & \frac{1}{a-2} & \frac{-2}{a-2} 
  \end{pmatrix}
  \xrightarrow{ R_{1} \leftarrow R_{1} + 2R_{2} + R_{3} }
\end{equation*}

\begin{equation*}
  \begin{pmatrix}
    1 & 0 & 0 & \frac{8-a}{5(a-2)} & -\frac{3}{5(a-2)} & \frac{a+4}{5(a-2)}\\
    0 & 1 &  0 & \frac{2(2+a)}{5(a-2)} & -\frac{4}{5(a-2)} & \frac{3a+2}{5(a-2)} \\
    0 & 0 &  1 & \frac{-2}{a-2} & \frac{1}{a-2} & \frac{-2}{a-2}
  \end{pmatrix}
\end{equation*}

Therefore,

\begin{equation*}
  A^{-1} = \begin{pmatrix}
    \frac{8-a}{5(a-2)} & -\frac{3}{5(a-2)} & \frac{a+4}{5(a-2)}\\
    \frac{2(2+a)}{5(a-2)} & -\frac{4}{5(a-2)} & \frac{3a+2}{5(a-2)} \\
    \frac{-2}{a-2} & \frac{1}{a-2} & \frac{-2}{a-2}
  \end{pmatrix}
   = \frac{1}{5(a-2)} \begin{pmatrix}
    8-a & -3 & a + 4\\
    2(2+a) & -4 & 3a+2 \\
     -10 & 5 & -10
  \end{pmatrix}
\end{equation*}

\textbf{c.} The solution to $A\mathbf{x} = \mathbf{b}$ is $\mathbf{x} = A^{-1}\mathbf{b}$.

\begin{equation*}
  \mathbf{x} = \frac{1}{5(a-2)} \begin{pmatrix}
    8-a & -3 & a + 4\\
    2(2+a) & -4 & 3a+2 \\
     -10 & 5 & -10
  \end{pmatrix} \begin{pmatrix} b \\ 3 \\ 3  \end{pmatrix} = \frac{1}{5(a-2)}\begin{pmatrix}
    b(8-a) - 9 + 3(a+4) \\
    2b(2+a) - 12 + 3(3a+2) \\
    -10b + 15 - 30
  \end{pmatrix} =
\end{equation*}

\begin{equation*}
   \frac{1}{5(a-2)}
  \begin{pmatrix}
    8b - ab - 9 + 3a + 12 \\
    4b + 2ab - 12 + 9a + 6 \\
    -10b - 15
  \end{pmatrix}  =
  \frac{1}{5(a-2)}
  \begin{pmatrix}
    8b - ab + 3a + 3 \\
    4b + 2ab - 6 + 9a \\
    -10b - 15
  \end{pmatrix}
\end{equation*}

Therefore, the solution is

\begin{equation*}
  \mathbf{x} =
  \begin{pmatrix}
    \frac{8b-ab+3a+3}{5(a-2)} & \frac{4b + 2ab - 6 + 9a}{5(a-2)} & -\frac{2b + 3}{a - 2}
  \end{pmatrix}^{T}
\end{equation*}

\section{Problem 7}

\textbf{a.} Partition matrix $A$ in such a way that the off-diagonal components are zero-matrices ($O_{n \times m}$).

\begin{equation*}
  A = \left(
  \begin{array}{c|c}
    A_{2\times2} & O_{2\times3} \\
    \hline
    O_{3\times2} & B_{3 \times 3}
  \end{array}
  \right)
\end{equation*},

where $A_{2 \times 2} =
\begin{pmatrix}
  1 & 2 \\
  3 & 5
\end{pmatrix}
$ and $B_{3 \times 3} =
\begin{pmatrix}
  2 & 0 & 0 \\
  0 & 7 & 8 \\
  0 & 5 & 6
\end{pmatrix}
$. The transpose of $A$ is:

\begin{equation*}
  A^{T} = \left(
  \begin{array}{c|c}
    A_{2\times2}^{T} & O_{3\times2}^{T} \\
    \hline
    O_{2\times3}^{T} & B_{3 \times 3}^{T}
  \end{array}
\right) =
\left(
  \begin{array}{c|c}
    A_{2\times2}^{T} & O_{2\times3} \\
    \hline
    O_{3\times2} & B_{3 \times 3}^{T}
  \end{array}
  \right)
\end{equation*}

Therefore, the square of $A^{T}$ is:

\begin{equation*}
  (A^{T})^{2} = \left(
  \begin{array}{c|c}
    A_{2\times2}^{T} & O_{2\times3} \\
    \hline
    O_{3\times2} & B_{3 \times 3}^{T}
  \end{array}
  \right) \left(
  \begin{array}{c|c}
    A_{2\times2}^{T} & O_{2\times3} \\
    \hline
    O_{3\times2} & B_{3 \times 3}^{T}
  \end{array}
\right) = \left(
  \begin{array}{c|c}
    (A_{2\times2}^{T})^{2} + O_{2\times3} O_{3\times2} & A_{2\times2}^{T} O_{2 \times 3} + O_{2 \times 3} B_{3\times3}^{T} \\ \hline
   O_{3\times2} A_{2\times2}^{T} + B_{3\times3}^{T} O_{3\times2} & O_{3\times2} O_{2\times3} + (B_{3\times3}^{T})^{2}
  \end{array}
\right)
\end{equation*}

\begin{equation*}
  = \left(
  \begin{array}{c|c}
    (A_{2\times2}^{T})^{2} + O_{2\times2} & O_{2 \times 3} + O_{2 \times 3} \\ \hline
   O_{3\times2} + O_{3\times2} & O_{3\times3} + (B_{3\times3}^{T})^{2}
  \end{array}
\right) = \left(
  \begin{array}{c|c}
    (A_{2\times2}^{T})^{2} & O_{2 \times 3} \\ \hline
   O_{3\times2} & (B_{3\times3}^{T})^{2}
  \end{array}
\right)
\end{equation*}

Now, evaluate each block.

\begin{equation*}
  (A_{2\times2}^{T})^{2} =
  \begin{pmatrix}
    1 & 3 \\
    2 & 5
  \end{pmatrix} \begin{pmatrix}
    1 & 3 \\
    2 & 5
  \end{pmatrix} =
  \begin{pmatrix}
    7 & 18 \\ 12 & 31
  \end{pmatrix}
\end{equation*}

\begin{equation*}
  (B_{3\times3}^{T})^{2} =
  \begin{pmatrix}
    2 & 0 & 0 \\
    0 & 7 & 5 \\
    0 & 8 & 6
  \end{pmatrix} \begin{pmatrix}
    2 & 0 & 0 \\
    0 & 7 & 5 \\
    0 & 8 & 6
  \end{pmatrix} =
  \begin{pmatrix}
    4 & 0 & 0 \\
    0 & 89 & 46 \\
    0 & 104 & 52 
  \end{pmatrix}
\end{equation*}

Substituting back into $(A^{T})^{2}$, we obtain:

\begin{equation*}
  (A^{T})^{2} =
  \begin{pmatrix}
    7 & 18 & 0 & 0 & 0 \\
    12 & 31 & 0 & 0 & 0 \\
    0 & 0 & 4 & 0 & 0 \\
    0 & 0 & 0 & 89 & 65 \\
    0 & 0 & 0 & 104 & 76
  \end{pmatrix}
\end{equation*}

\textbf{b.} Use the same partitioned $A$, calculate the inverse by using Gauss-Jordan elimination:

\begin{equation*}
  A = \left(
  \begin{array}{c|c}
    A_{2\times2} & O_{2 \times 3} \\ \hline
    O_{3\times2} & B_{3\times3}
  \end{array}
\right)
\end{equation*}

\begin{equation*}
  \left(
    \begin{array}{c|c|c|c}
      A_{2\times2} & O_{2\times3} & I_{2\times2} & O_{2\times3} \\ \hline
      O_{3\times2} & B_{3\times3} & O_{3\times2} & I_{3\times3}
    \end{array}
  \right)
  \xrightarrow{(R_{1} \leftarrow A_{2\times2}^{-1} R_{1}) \land (R_{2} \leftarrow B_{3\times3}^{-1}R_{2})}
    \left(
    \begin{array}{c|c|c|c}
      A_{2\times2}^{-1}A_{2\times2} & A_{2\times2}^{-1}O_{2\times3} & A_{2\times2}^{-1}I_{2\times2} & A_{2\times2}^{-1}O_{2\times3} \\ \hline
      B_{3\times3}^{-1}O_{3\times2} & B_{3\times3}^{-1}B_{3\times3} & B_{3\times3}^{-1}O_{3\times2} & B_{3\times3}^{-1}I_{3\times3}
    \end{array}
  \right) = 
\end{equation*}

\begin{equation*}
  \left(
    \begin{array}{c|c|c|c}
      I_{2\times2} & O_{2\times3} & A_{2\times2}^{-1} & O_{2\times3} \\ \hline
      O_{3\times2} & I_{3\times3} & O_{3\times2} & B_{3\times3}^{-1}
    \end{array}
  \right)
\end{equation*}

Therefore, $A^{-1} = \left(
    \begin{array}{c|c}
      A_{2\times2}^{-1} & O_{2\times3} \\ \hline
      O_{3\times2} & B_{3\times3}^{-1}
    \end{array}
  \right)$, which is expected since partioned matrix $A$ is diagonal. Now, Evaluate each block.

\begin{equation*}
  A_{2\times2}^{-1}:
  \begin{pmatrix}
    1 & 2 & 1 & 0 \\
    3 & 5 & 0 & 1
  \end{pmatrix} =
  \begin{pmatrix}
    1 & 2 & 1 & 0 \\
    0 & -1 & -3 & 1
  \end{pmatrix} = 
  \begin{pmatrix}
    1 & 2 & 1 & 0 \\
    0 & 1 & 3 & -1
  \end{pmatrix} =
  \begin{pmatrix}
    1 & 0 & -5 & 2 \\
    0 & 1 & 3 & -1
  \end{pmatrix} \Rightarrow A_{2\times2}^{-1} =
  \begin{pmatrix}
    -5 & 2 \\
    3 & -1
  \end{pmatrix}
\end{equation*}

\begin{equation*}
  B_{3\times3}^{-1}:
  \begin{pmatrix}
    2 & 0 & 0 & 1 & 0 & 0 \\
    0 & 7 & 8 & 0 & 1 & 0 \\
    0 & 5 & 6 & 0 & 0 & 1
  \end{pmatrix} =
  \begin{pmatrix}
    1 & 0 & 0 & 1/2 & 0 & 0 \\
    0 & 1 & 1 & 0 & 1/2 & -1/2 \\
    0 & 5 & 6 & 0 & 0 & 1
  \end{pmatrix} =
    \begin{pmatrix}
    1 & 0 & 0 & 1/2 & 0 & 0 \\
    0 & 1 & 0 & 0 & 3 & -4 \\
    0 & 0 & 1 & 0 & -5/2 & 7/2
    \end{pmatrix}
  \end{equation*}

  \begin{equation*}
    \Rightarrow B_{3\times3}^{-1} =
    \begin{pmatrix}
      1/2 & 0 & 0 \\
      0 & 3 & -4 \\
      0 & -5/2 & 7/2
    \end{pmatrix}    
  \end{equation*}

  Therefore,

  \begin{equation*}
A^{-1} =
\begin{pmatrix}
  -5 & 2 & 0 & 0 & 0 \\
  3 & -1 & 0 & 0 & 0 \\
  0 & 0 & 1/2 & 0 & 0 \\
  0 & 0 & 0 & 3 & -4 \\
  0 & 0 & 0 & -5/2 & 7/2
\end{pmatrix}
  \end{equation*}

  \textbf{c.} Calculate the determinant of $A$ using cofactor expansion along the first row. Then, the second and the third cofactors are irrelevant since they are multiplied with $0$'s.

  \begin{equation*}
    \begin{vmatrix}
      4 & 0 & 0 & -8 & 6 \\
      -3 & 1 & -1 & 6 & -2 \\
      1 & 4 & -4 & -2 & 8 \\
      3 & -2 & 2 & -6 & 7 \\
      -2 & 1 & -1 & 4 & -3
    \end{vmatrix} =
    4\begin{vmatrix}
      1 & -1 & 6 & -2 \\
      4 & -4 & -2 & 8 \\
      -2 & 2 & -6 & 7 \\
      1 & -1 & 4 & -3
    \end{vmatrix}
    +8\begin{vmatrix}
      -3 & 1 & -1 & -2 \\
      1 & 4 & -4 & 8 \\
      3 & -2 & 2 & 7 \\
      -2 & 1 & -1 & -3
    \end{vmatrix}
    -6 \begin{vmatrix}
      -3 & 1 & -1 & 6 \\
      1 & 4 & -4 & -2 \\
      3 & -2 & 2 & -6 \\
      -2 & 1 & -1 & 4
      \end{vmatrix}
  \end{equation*}

  First, analyze $det(M_{11})$. The first two columns can be all reduced to $\begin{pmatrix} 1 & -1 \end{pmatrix}$ and $\begin{pmatrix} 0 & 0 \end{pmatrix}$. This results in having different equations like $cx_{3} = d$ with different values of $d$ and nonzero values of $c$. This means that the system is inconsistent. Thus, $det(M_{11}) = 0$. In $M_{14}$, the same thing occurs with the second and third columns, so its determinant is also $0$. In $M_{15}$, this occurs with the first and fourth columns, so its determinent is also $0$. Combining all (partial) determinants, we get $det(A) = 0$ meaning that A is singular and, thus, not invertible.

\section{Problem 8}

\textbf{a.} Consider two matrices: $P_{2\times4}$ - prices of products in each shop, and $N_{3\times4}$ - number of products needed by each person. The matrices are:

\begin{equation*}
  P_{2\times4} = 
  \left(
    \begin{array}{c|c c c c}
      / & C & B_{r} & B_{e} & W \\ \hline
      S_{1} & 1.50 & 2.00 & 5.00 & 16.00 \\
      S_{2} & 1.00 & 2.50 & 4.50 & 17.00
    \end{array}
  \right) =
  \begin{pmatrix}
    \mathbf{p_{S_{1}}} \\
    \mathbf{p_{S_{2}}}
  \end{pmatrix}
\end{equation*}

\begin{equation*}
  N_{3\times4} =
  \left(
    \begin{array}{c|c c c c}
      / & C & B_{r} & B_{e} & W \\ \hline
      A & 6 & 5 & 3 & 1 \\
      D & 3 & 6 & 2 & 2 \\
      M & 3 & 4 & 3 & 1
    \end{array}
  \right) =
  \begin{pmatrix}
    \mathbf{n_{A}} \\
    \mathbf{n_{D}} \\
    \mathbf{n_{M}}
  \end{pmatrix}
\end{equation*}

Then, consider $P_{2\times4}N_{3\times4}^{T} = \begin{pmatrix}
    \mathbf{p_{S_{1}}} \\
    \mathbf{p_{S_{2}}}
  \end{pmatrix} \begin{pmatrix}
    \mathbf{n_{A}} &
    \mathbf{n_{D}} &
    \mathbf{n_{M}}
  \end{pmatrix} $ =
  $\begin{pmatrix}
    \mathbf{p_{S_{1}}} \mathbf{n_{A}}^{T} & \mathbf{p_{S_{1}}} \mathbf{n_{2}}^{T} & \mathbf{p_{S_{1}}} \mathbf{n_{3}}^{T} \\
    \mathbf{p_{S_{2}}} \mathbf{n_{A}}^{T} & \mathbf{p_{S_{2}}} \mathbf{n_{D}}^{T} & \mathbf{p_{S_{2}}} \mathbf{n_{M}}^{T}
  \end{pmatrix}$. Each inner\footnote{According to my notation, $\mathbf{p} \mathbf{n}^{T}$ is a row vector times a column vector which is a scalar (inner) product.} product $\mathbf{p_{i}}\mathbf{n_{j}}^{T}$ corresponds to the total price that person $j$ would pay in shop $i$.

\textbf{b.} Multiply the two matrices.

\begin{equation*}
  S = P_{2\times4}N_{3\times4}^{T} =
  \begin{pmatrix}
    1.50 & 2.00 & 5.00 & 16.00 \\
    1.00 & 2.50 & 4.50 & 17.00
  \end{pmatrix}
  \begin{pmatrix}
    6 & 3 & 3 \\
    5 & 6 & 4 \\
    3 & 2 & 3 \\
    1 & 2 & 1
  \end{pmatrix} =
  \begin{pmatrix}
    50.0 & 58.5 & 43.5 \\
    49.0 & 61.0 & 43.5  
  \end{pmatrix}
\end{equation*}

Therefore,

\begin{equation*}
  S =
  \left(
    \begin{array}{c|c c c}
      /     & A   & D & M \\ \hline
      S_{1} & 50.0 & 58.5 & 43.5 \\
      S_{2} & 49.0 & 61.0 & 43.5
    \end{array}
  \right)
\end{equation*}

represents the prices each person must pay in each shop.

\textbf{c.} For Anya, the best shop is $S_{2}$. For Dario, the best shop is $S_{1}$. For Marianne, the two shops are pricewise equivalent.

\section{Problem 9}

The number of atoms must be conserved on each side of the chemical reaction. Therefore,

\begin{equation*}
  \begin{cases}
    \textit{Carbon (C)}:   1a + 0b = 0c + 6d  \Rightarrow 1a + 0b - 0c - 6d = 0 \\
    \textit{Hydrogen (H)}: 0a + 2b = 0c + 12d \Rightarrow 0a + 2b - 0c - 12d = 0 \\
    \textit{Oxygen (O)}:   2a + 1b = 2c + 6d \Rightarrow  2a + 1b - 2c - 6d = 0
  \end{cases}
\end{equation*}

The augmented matrix corresponding to the linear system above is:
\begin{equation*}
  \begin{amatrix}{4}
    1 & 0 & 0 & -6 & 0\\
    0 & 2 & 0 & -12 & 0\\
    2 & 1 & -2 & -6 & 0
  \end{amatrix}
\end{equation*}

Solve using elementary row operations:

\begin{equation*}
  \begin{amatrix}{4}
    1 & 0 & 0 & -6 & 0\\
    0 & 2 & 0 & -12 & 0\\
    2 & 1 & -2 & -6 & 0
  \end{amatrix} =
  \begin{amatrix}{4}
    1 & 0 & 0 & -6 & 0\\
    0 & 1 & 0 & -6 & 0\\
    0 & 1 & -2 & 6 & 0
  \end{amatrix} =
   \begin{amatrix}{4}
     1 & 0 & 0 & -6 & 0\\
     0 & 1 & -2 & 6 & 0 \\
     0 & 0 & 2 & -12 & 0
   \end{amatrix} =
   \begin{amatrix}{4}
     1 & 0 & 0 & -6 & 0\\
     0 & 1 & 0 & -6 & 0 \\
     0 & 0 & 1 & -6 & 0
   \end{amatrix}
\end{equation*}

Therefore, $d$ is a free variable and $a = b = c = 6d$. Substitute the numbers back into the chemical reaction and cancel $d$ out because we need the minimum (and integer) number of molecules that take place in the photosynthesis reaction:

Answer: \ce{6CO2 + 6H2O -> 6O2 + C6H12O6}


\end{document}
