\subsection{Interpretation}
Figure \ref{fig:resonance} shows that experimental data from Table \ref{tab:data} follows the model given in Theory (section \ref{sec:theory}) by Eq.~\eqref{eq:full_motion_ampl} and Eq.~\eqref{eq:full_motion_phase}. There is a frequency ($\left< \omega_{res} \right> \approx 4.4 (rad/s)$), at which the amplitude is maximized; whereas the phase abruptly changes from $\pi/2$ to $-\pi/2$. According to the Theory (section \ref{sec:theory}), this particular frequency is called resonance frequency. Calculating this frequency from the two graphs yields $\omega_{amplitude} = 4.35 \pm 0.09 (rad/s)$ and $\omega_{phase} = 4.46 \pm 0.02 (rad/s)$ respectively. Averaging the result and calculating the error gives an actual resonance frequency of the system: $\omega_{res} = 4.4 \pm 0.1 (rad/s)$.

According to the hypothesis, damping factor can be evaluated from Eq.~\eqref{eq:freq_deps}: $\gamma_{indirect} = 1.32 \pm 0.8 (rad/s)$. This value coincides with the damping factor calculated directly ($\gamma_{direct}$).

Therefore, our hypothesis is correct and the relation between the damping factor and the resonance frequency is indeed $\omega_{res} = \sqrt{\omega^2 - 2\gamma^2}$, as given by Eq.~\eqref{eq:freq_deps}.

\subsection{Errors and Improvements}

In Figure \ref{fig:resonance}, some data points do not perfectly lie on the best-fit curves, or have large uncertainties. This behaviour has several sources.

Firstly, no-slipping assumption was made. However, during the experiment, the string slipped drastically for voltages greater than $4 (V)$, which disrupted the expected oscillatory motion leading to the uncertainties and deviations from the best-fit curves.

Besides, the springs underwent circular motion around the axis of oscillatory motion. This angular motion prevented the string from perfectly staying on the rotary motor, causing less friction and, hence, more slipping.

Also, the experiment was performed on a table that was connected with other teams. Sometimes, the table was slightly distorted from its position, causing additional (usually, insignificant) influence on the oscillating system.

Lastly, the string was made of the material that is easily deformed by large enough forces. Measurements with higher voltages ($\ge 4 (V)$) deformed the string, causing the system to deviate from the desired ideal oscillations.

Above mentioned experimental errors resulted in waveforms that are slightly assymetrical and that had tendency to shift away from the time axis. Besides, extrapolating data from those waveforms presented uncertainties, that are shown in green in Figure \ref{fig:resonance}. Phase was calculated by finding the maximum cross-correlation between the two waves (system oscillations and the driving force), which resulted in more uncertainties because of imperfect waves and insufficient accuracy of data obtained from motary sensors.

More accurate results could be obtained by following the next improvements. If the system was standing on a more stable base (table), no superfluous motion would disrupt the oscillatory motion. Also, rotary sensors can be replaced by more accurate devices. The strings and springs should ideally have only one degree of freedom: angular motion of the springs and strings should be restricted by using materials with a larger moment of inertia or adding low-friction walls around the strings, springs and the disk so that the slipping does not cause significant motion distortions.
